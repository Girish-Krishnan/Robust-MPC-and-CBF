\documentclass[11pt]{article}
\usepackage{amsmath, amssymb, amsthm, mathtools}
\usepackage{graphicx}
\usepackage{enumitem}
\usepackage{fullpage}
\usepackage{hyperref}

\title{Homework Assignment: Robust MPC and CBF-Based Safety Filtering for the Dubins Car Model}
\author{MAE 248: Safety for Autonomous Systems \\ Designer: Girish Krishnan}
\date{\today}

\begin{document}
\maketitle

\section{Introduction and Problem Statement}
In this assignment, you will implement several control strategies for the Dubins car model. The objectives are to:
\begin{enumerate}[label=(\alph*)]
    \item Implement a \textbf{Standard Model Predictive Control (MPC)} method for trajectory planning.
    \item Develop a \textbf{Robust MPC} formulation that accounts for bounded disturbances. You will consider different robust MPC methods (e.g., min-max formulation, tube-based MPC).
    \item Implement a \textbf{Safety Filter} using Control Barrier Functions (CBF) that modifies a nominal control input to ensure safety with respect to obstacles.
    \item Compare the performance of these controllers under various scenarios.
\end{enumerate}

The dynamics of the Dubins car are given by:
\begin{align}
    x_{k+1} &= x_k + v \cos(\theta_k)\, \Delta t, \label{eq:dyn_x} \\
    y_{k+1} &= y_k + v \sin(\theta_k)\, \Delta t, \label{eq:dyn_y} \\
    \theta_{k+1} &= \theta_k + u_k\, \Delta t, \label{eq:dyn_theta}
\end{align}
where $x_k, y_k \in \mathbb{R}$ denote the position, $\theta_k \in \mathbb{R}$ the heading, $v>0$ is a constant speed, and $u_k$ is the steering control input subject to 
\[
u_k \in \mathcal{U} = \{ u \in \mathbb{R} \mid -w_{\max} \le u \le w_{\max} \}.
\]

Obstacles in the environment are modeled as circles with centers $(o_x, o_y)$ and radii $r$. To guarantee safety, a margin $\delta > 0$ is added so that the safety condition for each obstacle is:
\[
(x - o_x)^2 + (y - o_y)^2 \ge (r + \delta)^2.
\]

\section{Part I: Standard MPC for the Dubins Car}
\textbf{Objective:} Develop an MPC formulation that minimizes a cost function over a finite horizon while ensuring dynamic feasibility and obstacle avoidance.

\subsection{Decision Variables and Dynamics Constraints}
Define the state trajectory over a prediction horizon $N$:
\[
X = \begin{bmatrix} x_0 \\ x_1 \\ \vdots \\ x_N \end{bmatrix}, \quad
Y = \begin{bmatrix} y_0 \\ y_1 \\ \vdots \\ y_N \end{bmatrix}, \quad
\Theta = \begin{bmatrix} \theta_0 \\ \theta_1 \\ \vdots \\ \theta_N \end{bmatrix},
\]
and the control inputs:
\[
U = \begin{bmatrix} u_0 \\ u_1 \\ \vdots \\ u_{N-1} \end{bmatrix}.
\]
The dynamics \eqref{eq:dyn_x}--\eqref{eq:dyn_theta} are enforced as equality constraints:
\[
\begin{aligned}
    x_{k+1} - x_k - v \cos(\theta_k)\, \Delta t &= 0, \\
    y_{k+1} - y_k - v \sin(\theta_k)\, \Delta t &= 0, \\
    \theta_{k+1} - \theta_k - u_k\, \Delta t &= 0, \quad \forall k=0,\ldots,N-1.
\end{aligned}
\]

\subsection{Cost Function}
Construct the cost function to penalize the final state error and control effort:
\[
J = \left( x_N - x_{\text{goal}} \right)^2 + \left( y_N - y_{\text{goal}} \right)^2 + \alpha \sum_{k=0}^{N-1} u_k^2,
\]
where $\alpha > 0$ is a small weight on the control effort.

\subsection{Obstacle Avoidance Constraints}
For each time step $k$ and each obstacle $(o_x, o_y, r)$, enforce:
\[
(x_k - o_x)^2 + (y_k - o_y)^2 \ge (r + \delta)^2.
\]

\subsection{Task}
\begin{enumerate}[label=\arabic*.]
    \item Formulate the full nonlinear programming problem (NLP) for the standard MPC.
    \item Use CasADi to define the decision variables, cost function, and constraints.
    \item Solve the NLP and return the first control input $u_0$.
\end{enumerate}

\section{Part II: Robust MPC Formulations}
\textbf{Objective:} Extend the standard MPC formulation to account for disturbances $d_k$ that satisfy $\| d_k \| \le d_{\max}$. Two common robust MPC approaches are:
\begin{enumerate}[label=(\alph*)]
    \item \textbf{Min-Max MPC:} Formulate a min-max optimization problem that minimizes the worst-case cost over all admissible disturbance sequences.
    \item \textbf{Tube-Based MPC:} Design a nominal MPC controller and an ancillary feedback controller to keep the system in a ``tube'' around the nominal trajectory.
\end{enumerate}

\subsection{Min-Max MPC}
\begin{enumerate}[label=\arabic*.]
    \item Write the disturbed dynamics:
    \[
    \begin{aligned}
    x_{k+1} &= x_k + v \cos(\theta_k)\, \Delta t + d^x_k,\\
    y_{k+1} &= y_k + v \sin(\theta_k)\, \Delta t + d^y_k,\\
    \theta_{k+1} &= \theta_k + u_k\, \Delta t,
    \end{aligned}
    \]
    with $\|d_k\| \le d_{\max}$.
    \item Develop the robust constraints by ensuring that the obstacle avoidance constraint holds for all disturbances.
    \item Formulate the min-max problem:
    \[
    \min_{U} \max_{d_k \in \mathcal{D}} J(U, d),
    \]
    and discuss possible relaxations or approximations.
\end{enumerate}

\subsection{Tube-Based MPC}
\begin{enumerate}[label=\arabic*.]
    \item Define a nominal system and an error dynamics for the deviation due to disturbances.
    \item Design a feedback controller to keep the error bounded, and tighten the constraints accordingly.
    \item Formulate the nominal MPC with tightened constraints to guarantee that the actual state remains within a tube around the nominal trajectory.
\end{enumerate}

\subsection{Task}
\begin{enumerate}[label=\arabic*.]
    \item Implement a robust MPC controller using one of the approaches above (or both for comparison).
    \item Analyze the performance of your robust MPC in simulation under different disturbance bounds $d_{\max}$.
    \item Compare your results with the standard MPC from Part I.
\end{enumerate}

\section{Part III: CBF-Based Safety Filtering}
\textbf{Objective:} Design a safety filter using Control Barrier Functions (CBF) that modifies a nominal control input $u_{\text{nom}}$ to ensure safety with respect to obstacles.

\subsection{Barrier Function and its Derivatives}
For each obstacle $(o_x, o_y, r)$, define the barrier function:
\[
h(x,y) = (x - o_x)^2 + (y - o_y)^2 - (r + \delta)^2.
\]
Its time derivative is:
\[
\dot{h}(x,y,\theta) = 2 v \left[ (x - o_x) \cos(\theta) + (y - o_y) \sin(\theta) \right].
\]
A simplified second derivative approximation can be written as:
\[
L_f^2 h(x,y,\theta) = 2v^2 + 2\lambda \dot{h}(x,y,\theta) + \lambda^2 h(x,y),
\]
and the control-dependent term:
\[
L_g L_f h(x,y,\theta) = 2v \left[ (y - o_y) \cos(\theta) - (x - o_x) \sin(\theta) \right].
\]

\subsection{CBF Constraint}
The safety condition is enforced by:
\[
L_f^2 h(x,y,\theta) + L_g L_f h(x,y,\theta) \, u \ge 0.
\]

\subsection{QP Formulation}
Formulate the following Quadratic Program (QP) to obtain the safe control input $u_{\text{safe}}$:
\[
\begin{aligned}
\min_{u} \quad & \frac{1}{2}(u - u_{\text{nom}})^2 \\
\text{s.t.} \quad & L_f^2 h(x,y,\theta) + L_g L_f h(x,y,\theta) \, u \ge 0, \quad \forall \text{ obstacles}, \\
& u \in [-w_{\max}, w_{\max}].
\end{aligned}
\]
You are required to solve this QP using CVXPY.

\subsection{Task}
\begin{enumerate}[label=\arabic*.]
    \item Derive the expressions for $h$, $\dot{h}$, $L_f^2h$, and $L_gL_fh$.
    \item Implement the QP in CVXPY that outputs $u_{\text{safe}}$.
    \item Validate your safety filter by comparing the trajectories generated using the nominal controller versus the safety-filtered control.
\end{enumerate}

\section{Part IV: Simulation and Comparative Analysis}
\textbf{Objective:} Evaluate and compare the performance of the Standard MPC, Robust MPC, and CBF-based Safety Filter under various scenarios.

\subsection{Simulation Setup}
\begin{enumerate}[label=\arabic*.]
    \item Define the simulation parameters: time step $\Delta t$, prediction horizon $N$, constant speed $v$, control bounds $w_{\max}$, and safety margin $\delta$.
    \item Set up initial conditions and goal positions.
    \item Consider multiple scenarios by varying:
    \begin{itemize}
        \item Disturbance bounds $d_{\max}$ for the robust MPC.
        \item Obstacle configurations.
        \item Presence or absence of disturbances in the simulation.
    \end{itemize}
\end{enumerate}

\subsection{Performance Metrics}
Compare the methods based on:
\begin{enumerate}[label=\arabic*.]
    \item \textbf{Goal Convergence:} Distance to goal over time.
    \item \textbf{Constraint Violation:} Minimum distance to obstacles.
    \item \textbf{Control Effort:} Magnitude and smoothness of the control input.
\end{enumerate}

\subsection{Task}
\begin{enumerate}[label=\arabic*.]
    \item Simulate each control method and plot the resulting trajectories.
    \item Provide a quantitative and qualitative comparison of the controllers.
    \item Discuss the trade-offs between performance and robustness for each method.
\end{enumerate}

\section{Submission Instructions}
\begin{itemize}
    \item Submit your annotated code files (in Python) along with a brief report (PDF) summarizing your results and observations.
    \item Include plots comparing the trajectories and control inputs for the different methods.
    \item Provide detailed explanations of your implementation choices and any assumptions made.
\end{itemize}

\section{Bonus}
For extra credit, implement both min-max MPC and tube-based MPC, and compare their performance in environments with high disturbance levels.

\end{document}
